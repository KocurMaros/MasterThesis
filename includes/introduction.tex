Interakcia človeka s robotickým systémom si vyžaduje nielen technickú presnosť a spoľahlivosť, ale čoraz častejšie aj schopnosť robota porozumieť svojmu ľudskému partnerovi na hlbšej, emocionálnej úrovni. V mnohých oblastiach nasadenia – od priemyslu cez zdravotníctvo až po domáce prostredie – sa ukazuje ako výhodné, ak robot dokáže prispôsobiť svoje správanie aktuálnemu emočnému stavu operátora. Takáto schopnosť môže zvýšiť efektivitu spolupráce, znížiť počet chýb a prispieť k celkovo prirodzenejšej a intuitívnejšej interakcii.

Táto diplomová práca sa zameriava na vytvorenie systému, ktorý bude schopný identifikovať emocionálne rozpoloženie človeka prostredníctvom analýzy výrazu tváre. Zámerom je, aby výsledný modul poskytoval robotickému systému relevantné informácie o emóciách používateľa v reálnom čase. Téma prepája oblasti počítačového videnia, umelej inteligencie a robotiky a reflektuje narastajúci význam emocionálnej inteligencie v moderných technológiách.
Interakcia človeka s robotom v dynamickom prostredí si čoraz viac vyžaduje pochopenie emocionálneho stavu operátora s cieľom optimalizovať komunikáciu a rozhodovacie procesy. Cieľom tejto práce je navrhnúť a implementovať modul, ktorý poskytuje robotickému systému emocionálnu spätnú väzbu a umožňuje mu zisťovať výrazy tváre operátora a odvodzovať jeho emocionálne stavy. S využitím kamery RGB s možnosťou integrácie kamery RGB-D bude systém využívať biometrické modely tváre a techniky rozpoznávania tváre na identifikáciu emócií v reálnom čase. Medzi kľúčové úlohy patrí analýza súčasných metód detekcie emócií výrazu tváre, štúdium princípov tvorby biometrických modelov tváre a implementácia robustného systému na detekciu emócií. Systém bude overený prostredníctvom testovania na simulovaných aj reálnych súboroch údajov. Okrem toho bude vyvinutý balík ROS2, ktorý zabezpečí bezproblémovú integráciu v rámci robotických systémov. Výsledky budú kriticky posúdené prostredníctvom experimentov s cieľom zabezpečiť presnosť a efektívnosť výkonu v reálnych aplikáciách.
\label{attachment:docker}

% Definícia štýlu pre Docker súbory
\lstdefinestyle{dockerstyle}{
    backgroundcolor=\color{backcolour},
    commentstyle=\color{codegreen},
    keywordstyle=\color{blue},
    numberstyle=\tiny\color{codegray},
    stringstyle=\color{codepurple},
    basicstyle=\ttfamily\footnotesize,
    breakatwhitespace=false,
    breaklines=true,
    captionpos=b,
    keepspaces=true,
    numbers=left,
    numbersep=5pt,
    showspaces=false,
    showstringspaces=false,
    showtabs=false,
    tabsize=2,
    morekeywords={FROM, RUN, CMD, LABEL, MAINTAINER, EXPOSE, ENV, ADD, COPY, ENTRYPOINT, VOLUME, USER, WORKDIR, ARG, ONBUILD, HEALTHCHECK, SHELL, version, services, build, image, ports, volumes, environment, depends_on, networks}
}

% Dockerfile
\begin{lstlisting}[style=dockerstyle, caption={Dockerfile pre rozpoznávanie emócií}, label={lst:dockerfile}]
# Use CUDA 11.2 and CUDNN 8 runtime with Ubuntu 20.04 as the base image
FROM nvidia/cuda:11.2.2-cudnn8-runtime-ubuntu20.04

# Set environment variables for non-interactive installation and Python path
ENV DEBIAN_FRONTEND=noninteractive
ENV PYTHON_VERSION=3.9
ENV PATH /usr/local/cuda/bin:$PATH

# Install dependencies and set up Python 3.9 environment
RUN apt-get update && \
        apt-get install -y --no-install-recommends \
        software-properties-common && \
        add-apt-repository ppa:deadsnakes/ppa && \
        apt-get update && \
        apt-get install -y --no-install-recommends \
        build-essential \
        curl \
        ca-certificates \
        python3.9 \
        python3.9-distutils \
        python3.9-dev \
        python3-pip \
        python3-setuptools \
        python3-venv \
        libopenblas-dev \
        libopencv-dev \
        && rm -rf /var/lib/apt/lists/*

# Upgrade pip and install required Python packages with specified versions
# Upgrade pip and install required Python packages with specified versions
RUN python3.9 -m pip install --upgrade pip && \
        python3.9 -m pip install \
        dlib==19.24.2 \
        matplotlib==3.8.3 \
        numpy==1.26.4 \
        opencv_python==4.9.0.80 \
        pandas==2.2.2 \
        Pillow==10.3.0 \
        retina_face==0.0.14 \
        seaborn==0.13.2 \
        torch==2.1.2 \
        torchvision==0.16.2 \
        tqdm==4.66.1 \
        urllib3==2.2.1 \
        jupyter
# Downgrade protobuf to resolve MediaPipe and TensorFlow compatibility issues
RUN python3.9 -m pip install protobuf==3.20.*

# Set Python 3.9 as the default Python version and link pip
RUN ln -sf /usr/bin/python3.9 /usr/bin/python && \
        ln -sf /usr/bin/pip3 /usr/bin/pip

# Set the default working directory inside the container
WORKDIR /workspace

# Copy local files to the container's workspace directory
COPY . .

# Expose the port for Jupyter Notebook
EXPOSE 8888

# Command to run Jupyter Notebook when the container starts
CMD ["jupyter", "notebook", "--ip=0.0.0.0", "--port=8888", "--no-browser", "--allow-root", "--NotebookApp.token=''"]
        
\end{lstlisting}

% docker-compose.yml
\begin{lstlisting}[style=dockerstyle, caption={docker-compose.yml pre systém rozpoznávania emócií}, label={lst:docker-compose}]
services:
tensorflow_gpu:
        build: .
        container_name: resEmoteNet
        runtime: nvidia
        environment:
        - NVIDIA_VISIBLE_DEVICES=all
        - NVIDIA_DRIVER_CAPABILITIES=compute,utility
        - PYTHONUNBUFFERED=1
        - CUDA_LAUNCH_BLOCKING=1  # Add this line
        deploy:
        resources:
        reservations:
        devices:
                - driver: nvidia
                count: all
                capabilities: [gpu]
        volumes:
        - .:/workspace
        working_dir: /workspace
        stdin_open: true
        tty: true
        ports:
        - "8888:8888"
        command: >
        bash -c "pip install notebook && 
        jupyter notebook --ip=0.0.0.0 --port=8888 --no-browser --allow-root"
      
\end{lstlisting}
\label{attachment:manual}

\subsection{Návod na trénovanie modelu}

Tento návod vás prevedie procesom trénovania modelu pre rozpoznávanie emócií pomocou Docker kontajnera a webového rozhrania.

\subsubsection{Požiadavky}
Na trénovanie modelu potrebujete:
\begin{itemize}
    \item Nainštalovaný Docker (\url{https://docs.docker.com/get-docker/})
    \item Nainštalovaný Docker Compose (\url{https://docs.docker.com/compose/install/})
    \item Minimálne 8 GB RAM a 4 GB GPU pamäte
    \item Dostatočné miesto na disku (minimálne 10 GB)
\end{itemize}

\subsubsection{Postup trénovania}

\begin{enumerate}
    \item \textbf{Príprava projektu}
    \begin{lstlisting}[basicstyle=\ttfamily\small]
    git clone https://github.com/KocurMaros/master_thesis_practical
    cd master_thesis_practical
    \end{lstlisting}
    
    \item \textbf{Zostavenie Docker kontajnera}\\
    V koreňovom adresári projektu spustite:
    \begin{lstlisting}[basicstyle=\ttfamily\small]
    docker-compose build
    \end{lstlisting}
    Tento príkaz zostaví Docker kontajner podľa definícií v súboroch Dockerfile a docker-compose.yml.
    
    \item \textbf{Spustenie kontajnera}\\
    Po úspešnom zostavení spustite kontajner:
    \begin{lstlisting}[basicstyle=\ttfamily\small]
    docker-compose up
    \end{lstlisting}
    
    \item \textbf{Prístup k webovému rozhraniu}\\
    Po spustení kontajnera otvorte webový prehliadač a prejdite na adresu:
    \begin{lstlisting}[basicstyle=\ttfamily\small]
    http://localhost:5000
    \end{lstlisting}
    
    \item \textbf{Spustenie trénovania}\\
    Vo webovom rozhraní:
    \begin{itemize}
        \item Vyberte dataset, ktorý chcete použiť na trénovanie
        \item Nastavte hyperparametre (počet epoch, batch size, learning rate)
        \item Kliknite na tlačidlo "Spustiť trénovanie"
        \item Sledujte priebeh trénovania a výsledné metriky
    \end{itemize}
    
    \item \textbf{Export natrénovaného modelu}\\
    Po ukončení trénovania si môžete stiahnuť natrénovaný model kliknutím na tlačidlo "Exportovať model".
\end{enumerate}

\subsection{Návod na testovanie s ROS2}

Tento návod popisuje, ako spustiť systém rozpoznávania emócií pomocou ROS2 Humble na Ubuntu 22.04.

\subsubsection{Požiadavky}
Na testovanie potrebujete:
\begin{itemize}
    \item Ubuntu 22.04 LTS
    \item ROS2 Humble (\url{https://docs.ros.org/en/humble/Installation.html})
    \item Kamera kompatibilná s ROS2
    \item Natrénovaný model (výstup z procesu trénovania)
\end{itemize}

\subsubsection{Inštalácia a zostavenie}
\begin{enumerate}
    \item \textbf{Príprava ROS2 workspace}
    \begin{lstlisting}[basicstyle=\ttfamily\small]
    mkdir -p ~/ros2_ws/src
    cd ~/ros2_ws/
    cp -r /path/to/your/master_thesis_practical/facial_expression ~/ros2_ws/src/
    \end{lstlisting}
    
    \item \textbf{Zostavenie balíka}
    \begin{lstlisting}[basicstyle=\ttfamily\small]
    colcon build --symlink-install --packages-select emotion_recognition
    \end{lstlisting}
    
    \item \textbf{Načítanie prostredia ROS2}
    \begin{lstlisting}[basicstyle=\ttfamily\small]
    source install/setup.bash
    \end{lstlisting}
\end{enumerate}

\subsubsection{Spustenie systému}
\begin{enumerate}
    \item \textbf{Spustenie video streamu}\\
    Najprv spustite skript pre získavanie videa z kamery:
    \begin{lstlisting}[basicstyle=\ttfamily\small]
    ./scripts/video_stream.sh
    \end{lstlisting}
    
    \item \textbf{Spustenie rozpoznávania emócií}\\
    Po spustení video streamu spustite rozpoznávanie emócií:
    \begin{lstlisting}[basicstyle=\ttfamily\small]
    ./scripts/run_facial_expression.sh
    \end{lstlisting}
    Tento skript aktivuje:
    \begin{itemize}
        \item Detekciu tváre
        \item Predikčný node pre rozpoznávanie emócií
        \item Webový server pre vizualizáciu výsledkov
    \end{itemize}
    
    \item \textbf{Prístup k výsledkom}\\
    Výsledky rozpoznávania emócií sú dostupné:
    \begin{itemize}
        \item Vo webovom rozhraní: http://localhost:8080
        \item V ROS2 topicoch:
        \begin{lstlisting}[basicstyle=\ttfamily\small]
        ros2 topic echo /emotion_prediction    # výsledok predikcie
        ros2 topic echo /face_detection        # detekované tváre
        \end{lstlisting}
    \end{itemize}
\end{enumerate}

\subsubsection{Riešenie problémov}
\begin{itemize}
    \item \textbf{Problém s kamerou:} Skontrolujte, či je kamera rozpoznaná systémom pomocou príkazu \texttt{v4l2-ctl --list-devices}
    \item \textbf{Problém s ROS2 nodmi:} Skontrolujte stav nodov pomocou \texttt{ros2 node list}
    \item \textbf{Chyba predikcie:} Uistite sa, že cesta k modelu v konfiguračnom súbore je správna
\end{itemize}

\subsubsection{Ukončenie systému}
Na ukončenie všetkých bežiacich nodov stlačte \textbf{Ctrl+C} v termináloch, kde sú spustené skripty, alebo použite:
\begin{lstlisting}[basicstyle=\ttfamily\small]
ros2 lifecycle set /emotion_recognition_node shutdown
\end{lstlisting}
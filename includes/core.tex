\section{Úvod}
Únava je bežnou súčasťou ľudského života a môže mať významný vplyv na našu bezpečnosť, produktivitu a celkovú pohodu. Obzvlášť v náročných pracovných prostrediach, ako je doprava alebo priemysel, môže únava viesť k vážnym následkom, vrátane nehôd a zranení. Cieľom tejto práce je vyvinúť systém na automatickú detekciu únavy na základe analýzy výrazov tváre. Takýto systém by mohol byť využitý v rôznych oblastiach, od monitorovania vodičov až po zlepšenie pracovného prostredia v priemyselných podnikoch.
\subsection{Motivácia}
\textit{Chcem tu este hodit ake typy emocii sa daju rozpoznat a preco je to dolezite.}
\textbf{Prečo je analýza emócií dôležitá v kontexte interakcie človeka s robotom?}

Únava, najmä v kritických profesiách a každodenných činnostiach, predstavuje značné riziko pre jednotlivcov aj spoločnosť. Detekcia únavy je kľúčová, pretože môže zabrániť nehodám, zvýšiť produktivitu a celkovo prispieť k zlepšeniu kvality života. Pre ľudí v náročných pracovných prostrediach, ako sú vodiči, operátori strojov alebo zdravotnícki pracovníci, môže únava spôsobiť významný pokles výkonnosti a zvýšenie rizika chýb, ktoré môžu mať fatálne dôsledky.
\subsection{Bezpečnosť}
\textbf{Doprava: Zníženie počtu nehôd spôsobených únavou vodičov}

Jedným z najdôležitejších dôvodov pre detekciu únavy je prevencia dopravných nehôd. Únava vodiča spôsobuje spomalenie reakčných časov, zníženú koncentráciu a vyššiu pravdepodobnosť mikrospánku. Podľa štatistík je únava zodpovedná za približne 20 \% všetkých dopravných nehôd. Systémy detekcie únavy, ktoré analyzujú fyziologické alebo behaviorálne signály vodiča, môžu v reálnom čase varovať pred rizikom mikrospánku alebo strate koncentrácie a tak potenciálne zachrániť ľudské životy a znížiť škody na majetku.

\textbf{Priemysel: Zníženie rizika úrazov v náročných pracovných prostrediach}

V priemyselných odvetviach, kde sú pracovníci vystavení vysokému fyzickému a psychickému zaťaženiu (napr. ťažký priemysel, nočné smeny, manipulácia s nebezpečnými materiálmi), únava môže viesť k chybám, ktoré môžu mať vážne následky. Monitorovanie únavy pomocou technológií by mohlo byť účinným spôsobom, ako predchádzať pracovným úrazom, zvýšiť bezpečnosť na pracovisku a znížiť počet nehôd spôsobených ľudskou chybou.
\subsection{Produktivita}
\textbf{Zvýšenie efektivity a výkonnosti}

Únava má priamy vplyv na zníženie koncentrácie, rýchlosti rozhodovania a celkového výkonu. Zamestnanci, ktorí pracujú unavení, sú menej produktívni, robia viac chýb a potrebujú viac času na vykonanie úloh. Systémy na monitorovanie únavy môžu pomôcť identifikovať okamihy, keď je výkon pracovníkov znížený, a navrhnúť prestávky alebo zmeny v pracovnom režime. Tým sa môže optimalizovať pracovný čas a zvýšiť celková efektivita pracovného procesu.

\textbf{Optimalizácia pracovného času}

Moderné technológie umožňujú monitorovanie úrovne únavy v reálnom čase a poskytujú zamestnávateľom a manažérom možnosť upraviť pracovné zaťaženie podľa aktuálnej úrovne únavy jednotlivých zamestnancov. Takáto optimalizácia môže znížiť riziko chýb, zvýšiť efektivitu a zároveň podporiť lepšiu pracovnú pohodu zamestnancov.

\subsection{Ďalšie aspekty}

\textbf{Zdravie prevencia zdravotných problémov}
People sometimes are on medications that create drowsiness or have physical ailments that cause these issues.

Chronická únava môže viesť k rôznym zdravotným problémom, ako sú depresia, úzkosť, oslabenie imunitného systému a zvýšené riziko kardiovaskulárnych ochorení. Monitorovanie únavy a zavádzanie preventívnych opatrení, ako sú pravidelné prestávky alebo úprava pracovných podmienok, môžu pomôcť predchádzať týmto problémom a zlepšiť celkové zdravie jednotlivcov.

Dostatok odpočinku je nevyhnutný pre udržanie duševnej a fyzickej pohody. Chronická únava znižuje kvalitu života, ovplyvňuje medziľudské vzťahy a môže viesť k problémom v osobnom a profesionálnom živote. Systémy na detekciu únavy môžu pomôcť jednotlivcom lepšie si plánovať odpočinok a dosiahnuť rovnováhu medzi pracovným a osobným životom.

\textbf{Sociálne dôsledky únavy}

Únava ovplyvňuje nielen pracovný výkon, ale aj medziludské vzťahy. Unavení ľudia majú často problémy s komunikáciou, trpia podráždenosťou a môžu sa vyhýbať spoločenským interakciám. Chronická únava môže viesť k sociálnej izolácii, čo má negatívny vplyv na kvalitu života.

\textbf{Ekonomické dôsledky únavy}

Únava má významné ekonomické dôsledky. Nehody spôsobené únavou vedú k veľkým finančným stratám v dôsledku škôd na majetku, zdravotných výdavkov a zníženej produktivity. Únava na pracovisku znižuje výkon a zvyšuje počet chýb, čo môže viesť k zníženiu kvality produktov a služieb, a tým aj k strate zákazníkov.
\subsection{Ciele práce}
Cieľom práce je vytvoriť systém, ktorý bude schopný rozpoznať emócie v reálnom čase.

\section{Štruktúra práce}
Práca je rozdelená do niekoľkých kapitol. V prvej kapitole sa zameriame na analýzu emócií. V druhej kapitole sa zameriame na analýzu dát. V tretej kapitole sa zameriame na návrh riešenia. V štvrtej kapitole sa zameriame na implementáciu riešenia. V piatej kapitole sa zameriame na vyhodnotenie riešenia. V šiestej kapitole sa zameriame na záver.

\section{Teoretické základy}
\subsection{Emócie a ich prejav}
Definícia emócií, univerzálne emócie, kultúrne rozdiely v prejave emócií, výrazy tváre ako indikátor emócií.
\subsection{Analýza obrazu}
Základné pojmy z oblasti analýzy obrazu, detekcia tváre, extrakcia príznakov, klasifikácia.
\subsection{Biometria}
Princípy biometrických systémov, identifikácia vs. verifikácia.

\section{Existujúce metody analýzy emócií}
\subsection{Ručne značenie}
\subsection{Automatická analýza emócií}

\section{Návrh riešenia}
\subsection{Architektúra systému}
\subsection{Výber dát}
\subsection{Extrakcia príznakov}
\subsection{Klasifikácia}
\subsection{Vyber hyperparametrov}

\section{Implementácia riešenia}
\subsection{Výber nástrojov}Programovací jazyk, knižnice (OpenCV, TensorFlow, PyTorch).
\subsection{Implementácia jednotlivých komponentov}Podrobný popis implementácie.
\subsection{Vizualizácia výsledkov}Vizualizácia výsledkov analýzy. Grafy, tabuľky.

\section{Exprerimenty a vyhodnotenie}
\subsection{Dátová sada} Popis použitého dataset-u (veľkosť, rozdelenie tried, kvalita).
\subsection{Metriky} Výber vhodných metrik (presnosť, úplnosť, F1-skóre, ROC krivka).
\subsection{Výsledky} Vyhodnotenie výsledkov experimentov. Prehľadné zhrnutie výsledkov, porovnanie s inými prácami.
\subsection{Analýza výsledkov} Analýza výsledkov, príčiny chýb, možné zlepšenia.

\section{Implementácia v ROS2}
\subsection{Konverzia modelu}Konverzia trénovaného modelu do formátu vhodného pre ROS2.
\subsection{Integrácia do robotického systému}Popis integrácie do ROS2, komunikácia s ostatnými modulmi.

\section{Záver}
\subsection{Zhodnotenie práce}Zhodnotenie dosiahnutých výsledkov.
\subsection{Obmedzenia práce}Obmedzenia práce, možné zlepšenia.
\subsection{Budúce smerovanie}Možné smerovanie ďalšej práce.

\section{Doplnujece poznamky }
Literatúra: Pravidelne citujte relevantnú literatúru.
Obrázky a diagramy: Používajte obrázky a diagramy na ilustráciu komplexných konceptov.
Kód: Ak je to možné, pridajte ukážky kódu.
Tabuľky: Používajte tabuľky na porovnanie výsledkov.
Táto štruktúra poskytuje komplexný rámec pre vašu prácu. Môžete ju prispôsobiť podľa svojich konkrétnych potrieb a zistení.